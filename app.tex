% ============================================
% CHƯƠNG: XÂY DỰNG VÀ TRIỂN KHAI ỨNG DỤNG
% ============================================
\setlength{\parindent}{0pt}

\section{Xây dựng và Triển khai Ứng dụng}

Chương này trình bày kiến trúc hệ thống, giao diện người dùng và quy trình triển khai ứng dụng dự đoán đánh giá sản phẩm từ bình luận tiếng Việt trên nền tảng Hugging Face Spaces.

% ============================================
% PHẦN 1: KIẾN TRÚC HỆ THỐNG
% ============================================
\subsection{Kiến trúc Hệ thống}

\subsubsection{Mô hình tổng quan}

Hệ thống được thiết kế theo mô hình Client-Server với kiến trúc phân lớp, bao gồm bốn thành phần chính được mô tả trong Bảng \ref{tab:kien_truc}.

\begin{table}[H]
\centering
\caption{Các thành phần trong kiến trúc hệ thống}
\label{tab:kien_truc}
\begin{tabular}{|l|l|p{7cm}|}
\hline
\textbf{Lớp} & \textbf{Công nghệ} & \textbf{Chức năng} \\
\hline
Frontend & HTML, TailwindCSS, JavaScript & Hiển thị giao diện, gửi request qua Fetch API \\
\hline
API & FastAPI & Xử lý HTTP request, định tuyến đến các service \\
\hline
Service & Python & Xử lý logic nghiệp vụ: xác thực, dự đoán, visualization \\
\hline
Database & PostgreSQL & Lưu trữ thông tin người dùng và lịch sử dự đoán \\
\hline
\end{tabular}
\end{table}

Luồng xử lý dữ liệu diễn ra như sau: người dùng nhập bình luận trên giao diện web, JavaScript gửi HTTP POST request đến Backend API. Sau khi xác thực JWT token, hệ thống chuyển request đến ML Service để thực hiện dự đoán. Kết quả được trả về dạng JSON và hiển thị trên giao diện mà không cần tải lại trang.

\subsubsection{Cơ chế Lazy Loading}

Hệ thống áp dụng cơ chế Lazy Loading cho model PhoBERT nhằm tối ưu tài nguyên. Model chỉ được tải vào bộ nhớ khi có request dự đoán đầu tiên, thay vì tải ngay khi khởi động server. Bảng \ref{tab:lazy_loading} trình bày các lợi ích của cơ chế này.

\begin{table}[H]
\centering
\caption{Lợi ích của cơ chế Lazy Loading}
\label{tab:lazy_loading}
\begin{tabular}{|l|p{9cm}|}
\hline
\textbf{Lợi ích} & \textbf{Mô tả} \\
\hline
Khởi động nhanh & Server khởi động trong vài giây thay vì 30-60 giây \\
\hline
Tiết kiệm RAM & Model (~500MB) chỉ chiếm bộ nhớ khi thực sự cần thiết \\
\hline
Tối ưu Cold Start & Phù hợp với môi trường container và serverless \\
\hline
\end{tabular}
\end{table}

Model PhoBERT được lưu trữ trên Hugging Face Hub và tải về thông qua thư viện huggingface\_hub. Tokenizer được tải từ repository gốc vinai/phobert-base.

% ============================================
% PHẦN 2: GIAO DIỆN VÀ CHỨC NĂNG
% ============================================
\subsection{Giao diện và Chức năng}

\subsubsection{Màn hình Dashboard}

Dashboard là giao diện chính của ứng dụng, cho phép người dùng thực hiện dự đoán đánh giá. Giao diện được thiết kế responsive với TailwindCSS, hỗ trợ cả chế độ sáng và tối. Hình \ref{fig:giao_dien_chinh} minh họa màn hình Dashboard.

\begin{figure}[H]
    \centering
    \includegraphics[width=0.85\textwidth]{images/giao_dien_chinh.png}
    \caption{Màn hình Dashboard với form nhập liệu và kết quả dự đoán}
    \label{fig:giao_dien_chinh}
\end{figure}

Các thành phần chính bao gồm: thanh điều hướng, vùng nhập liệu với hai chế độ (Single Comment và Upload CSV), và khu vực hiển thị kết quả.

\subsubsection{Chức năng Dự đoán Đơn lẻ}

Chức năng này cho phép người dùng nhập một bình luận tiếng Việt và nhận kết quả dự đoán. Kết quả hiển thị bao gồm: rating dự đoán (1-5 sao), độ tin cậy, bình luận với từ khóa được highlight, và danh sách từ khóa tích cực/tiêu cực. Hình \ref{fig:ket_qua_du_doan} minh họa kết quả dự đoán.

\begin{figure}[H]
    \centering
    \includegraphics[width=0.85\textwidth]{images/ket_qua_du_doan.png}
    \caption{Kết quả dự đoán với Rating, Confidence và Keyword Highlighting}
    \label{fig:ket_qua_du_doan}
\end{figure}

\subsubsection{Chức năng Dự đoán Hàng loạt}

Tính năng Batch Prediction cho phép upload file CSV chứa nhiều bình luận để dự đoán đồng thời. File CSV cần có cột ``Comment'' chứa nội dung bình luận. Bảng \ref{tab:batch_result} mô tả các thành phần trong kết quả.

\begin{table}[H]
\centering
\caption{Các thành phần trong kết quả Batch Prediction}
\label{tab:batch_result}
\begin{tabular}{|l|p{8cm}|}
\hline
\textbf{Thành phần} & \textbf{Mô tả} \\
\hline
Rating Distribution & Biểu đồ phân bố số lượng bình luận theo từng mức rating \\
\hline
Word Cloud & Đám mây từ khóa thể hiện tần suất xuất hiện của các từ \\
\hline
N-gram Analysis & Thống kê cụm từ phổ biến (1-gram, 2-gram, 3-gram) \\
\hline
Results Table & Bảng chi tiết kết quả dự đoán cho từng bình luận \\
\hline
Export & Xuất kết quả ra file CSV hoặc PDF report \\
\hline
\end{tabular}
\end{table}

% ============================================
% PHẦN 3: TRIỂN KHAI
% ============================================
\subsection{Triển khai trên Hugging Face Spaces}

\subsubsection{Giới thiệu Nền tảng}

Hugging Face Spaces là nền tảng hosting miễn phí cho các ứng dụng Machine Learning, hỗ trợ Docker SDK với 16GB RAM. Bảng \ref{tab:hf_spaces} trình bày các ưu điểm của nền tảng này.

\begin{table}[H]
\centering
\caption{Ưu điểm của Hugging Face Spaces}
\label{tab:hf_spaces}
\begin{tabular}{|l|p{9cm}|}
\hline
\textbf{Ưu điểm} & \textbf{Mô tả} \\
\hline
Miễn phí & Cung cấp CPU với 16GB RAM, đủ để chạy model PhoBERT \\
\hline
Docker Support & Hỗ trợ Docker SDK cho các ứng dụng phức tạp \\
\hline
Git-based & Triển khai đơn giản bằng git push \\
\hline
Tích hợp Hub & Dễ dàng tải model từ Hugging Face Hub \\
\hline
\end{tabular}
\end{table}

\subsubsection{Quy trình Triển khai}

Quy trình triển khai ứng dụng gồm ba bước chính:

\begin{enumerate}
    \item \textbf{Tạo Space}: Truy cập Hugging Face, tạo Space mới với Docker SDK và chọn CPU Basic (16GB RAM).
    \item \textbf{Cấu hình biến môi trường}: Thiết lập DATABASE\_URL (PostgreSQL connection string) và SECRET\_KEY (JWT secret key) trong phần Settings.
    \item \textbf{Push code}: Clone repository của Space, copy source code và push lên để trigger quá trình build.
\end{enumerate}

Ứng dụng sử dụng Dockerfile để định nghĩa môi trường runtime, bao gồm Python 3.10 và cấu hình port 7860 theo yêu cầu của Hugging Face Spaces.

\subsubsection{Kết quả Triển khai}

Ứng dụng đã được triển khai thành công và hoạt động ổn định tại địa chỉ:

\begin{center}
\url{https://vtdung23-predict-rating.hf.space}
\end{center}

Người dùng có thể truy cập Dashboard tại đường dẫn trên hoặc API Documentation (Swagger) tại \url{https://vtdung23-predict-rating.hf.space/docs}. Hình \ref{fig:hf_deployment} minh họa giao diện ứng dụng sau khi triển khai.

\begin{figure}[H]
    \centering
    \includegraphics[width=0.85\textwidth]{images/hf_deployment.png}
    \caption{Ứng dụng đã được triển khai trên Hugging Face Spaces}
    \label{fig:hf_deployment}
\end{figure}
